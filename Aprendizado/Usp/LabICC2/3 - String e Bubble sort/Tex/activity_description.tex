% activity_description.tex
% A simplified, clean layout for the activity description.

\vspace{0.5cm} % Adjust this space to position the section vertically on the page

% --- Title ---
% We use \parbox to keep the title and the rule together.
\noindent\parbox{\linewidth}{
    {\color{titleblue}\LARGE\bfseries A. Descrição da atividade}
    \vspace{2mm} % Space between title and the line
    
    % A thin, lighter-blue horizontal rule
    %\noindent\color{titleblue!50}\rule{\linewidth}{0.6pt}
}

\bigskip % Space after the title section

% --- Body Text ---
%\begin{raggedright} % Aligns text to the left for better readability
Com os inscritos de cada comunidades separados, agora é necessário realizar a organização deles. \\
Para isso, os inscritos de cada comunidade devem ser ordenados pelo tamanho dos seus nomes (número de caracteres,desconsiderando espaços em branco). 
Você deve implementar o algoritmo \textbf{Bubble Sort}, para ordenar os inscritos de cada comunidade

\medskip % A little vertical space

\noindent\textbf{No relatório deve estar presente:}
\begin{itemize}
    \setlength\itemsep{0.2em} % Adjust spacing between list items
    \item Descrição sucinta do funcionamento do algoritmo Bubble Sort.
    \item Código da implementação da sua solução
    \item Análise crítica de desempenho do algoritmo, discutindo:
    \begin{itemize}
        \item Tempo de execução medido no Run.codes \textbf{(CPU time)}.
        \item Número de comparações e trocas.
        \item Três cenários distintos:
        \begin{itemize}
            \item \textbf{Melhor caso:} lista já ordenada
            \item \textbf{Pior caso:} lista ordenada de forma inversa.
            \item \textbf{Caso médio:} lista em ordem aleatória.
        \end{itemize}
        \item Discussão sobre simplicidade do código e uso de funções auxiliares
    \end{itemize}
\end{itemize}

%\end{raggedright}
