% covermacros.tex
% Colors, user macros and TikZ shadow routine

% load TikZ (we put it here so main.tex is lean)
\usepackage{tikz}
\usetikzlibrary{shadows,calc}

% ---------- user-editable fields ----------
\newcommand{\titleEntrega}{Entrega 3.2}
\newcommand{\studentname}{Christyan Paniago Nantes}
\newcommand{\nusp}{15635906}
\newcommand{\duedate}{25/09/2025}
\newcommand{\institute}{Instituto De Ciências Matemáticas e de Computação}
\newcommand{\department}{Departamento de Ciências de Computação}
\newcommand{\course}{SCC0220 - Laboratório de Introdução à Ciência da Computação II}
\newcommand{\classinfo}{Turma BCC -- Prof. Jean Roberto Ponciano}
% ------------------------------------------

% colors (made here so cover.tex can use them)
\definecolor{uspblue}{RGB}{0,174,239}
\definecolor{line}{RGB}{136,184,227}
\definecolor{titleblue}{RGB}{15,111,198} % ADDED: New color for the activity title

% ---------- shadow parameters (tweak if needed) ----------
% xshift (right), yshift (down) for the shadow
\def\shadowshift{3pt,-3pt}    % Tighter offset
\def\shadowradius{6pt}        % Sharper blur
\colorlet{innercolor}{black!25} 
\colorlet{outercolor}{black!3}  
% -----------------------------------------------------

% drawshadow: draws a soft rounded shadow for a node name #1
\newcommand\drawshadow[1]{%
  \begin{pgfonlayer}{shadow}%
    \shade[inner color=innercolor, outer color=outercolor]
      ($(#1.south west)+(\shadowshift)+(\shadowradius/2,\shadowradius/2)$) circle (\shadowradius);
    \shade[inner color=innercolor, outer color=outercolor]
      ($(#1.north west)+(\shadowshift)+(\shadowradius/2,-\shadowradius/2)$) circle (\shadowradius);
    \shade[inner color=innercolor, outer color=outercolor]
      ($(#1.south east)+(\shadowshift)+(-\shadowradius/2,\shadowradius/2)$) circle (\shadowradius);
    \shade[inner color=innercolor, outer color=outercolor]
      ($(#1.north east)+(\shadowshift)+(-\shadowradius/2,-\shadowradius/2)$) circle (\shadowradius);
    \shade[top color=innercolor, bottom color=outercolor]
      ($(#1.south west)+(\shadowshift)+(\shadowradius/2,-\shadowradius/2)$)
      rectangle
      ($(#1.south east)+(\shadowshift)+(-\shadowradius/2,\shadowradius/2)$);
    \shade[left color=innercolor, right color=outercolor]
      ($(#1.south east)+(\shadowshift)+(-\shadowradius/2,\shadowradius/2)$)
      rectangle
      ($(#1.north east)+(\shadowshift)+(\shadowradius/2,-\shadowradius/2)$);
    \shade[bottom color=innercolor, top color=outercolor]
      ($(#1.north west)+(\shadowshift)+(\shadowradius/2,-\shadowradius/2)$)
      rectangle
      ($(#1.north east)+(\shadowshift)+(-\shadowradius/2,\shadowradius/2)$);
    \shade[right color=innercolor, left color=outercolor]
      ($(#1.south west)+(\shadowshift)+(-\shadowradius/2,\shadowradius/2)$)
      rectangle
      ($(#1.north west)+(\shadowshift)+(\shadowradius/2,-\shadowradius/2)$);
    
    % Replaced the solid \fill with a \shade to create a soft gradient core
    \shade[top color=innercolor, bottom color=innercolor]
      ($(#1.south west)+(\shadowshift)+(\shadowradius/2,\shadowradius/2)$)
      rectangle
      ($(#1.north east)+(\shadowshift)-(\shadowradius/2,\shadowradius/2)$);
  \end{pgfonlayer}%
}%

% declare the shadow layer (so it draws under main content)
\pgfdeclarelayer{shadow}
\pgfsetlayers{shadow,main}